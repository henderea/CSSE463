% Template for ICME-2010 paper; to be used with:
%          spconf.sty  - ICASSP/ICIP LaTeX style file, and
%          IEEEbib.bst - IEEE bibliography style file.
% --------------------------------------------------------------------------
\documentclass{article}
\usepackage{spconf_ICME,amsmath,epsfig,fancyhdr}
\setlength{\paperwidth}{215.9mm} \setlength{\hoffset}{-9.7mm}
\setlength{\oddsidemargin}{0mm} \setlength{\textwidth}{184.3mm}
\setlength{\columnsep}{6.3mm} \setlength{\marginparsep}{0mm}
\setlength{\marginparwidth}{0mm} \setlength{\paperheight}{279.4mm}
\setlength{\voffset}{-7.4mm} \setlength{\topmargin}{0mm}
\setlength{\headheight}{0mm} \setlength{\headsep}{0mm}
\setlength{\topskip}{0mm} \setlength{\textheight}{235.2mm}
\setlength{\footskip}{12.4mm} \setlength{\parindent}{1pc}


\ICMEfinalcopy % *** Uncomment this line for the final submission
\def\ICMEPaperID{}
\def\httilde{\mbox{\tt\raisebox{-.5ex}{\symbol{126}}}}

% Pages are numbered in submission mode, and unnumbered in camera-ready
\ifICMEfinal\pagestyle{empty}\fi


\begin{document}\sloppy

% Title.
% ------
\title{AUTOMATICALLY DISTINGUISHING BETWEEN SUNSET AND NON-SUNSET SCENES USING MACHINE LEARNING}
%
% Single address.
% ---------------
\name{Nicholas Kamper and Eric Henderson}
\address{Rose-Hulman Institute of Technology \\
Email: kampernj@rose-hulman.edu and henderea@rose-hulman.edu}


\maketitle
% insert page header and footer here for IEEE PDF Compliant
\thispagestyle{fancy} \fancyhead{} \lhead{}
\renewcommand{\headrulewidth}{0pt}
\renewcommand{\footrulewidth}{0pt}




%
\begin{abstract}
Scene detection is a complicated problem with many commercial applications in 
consumer products. For this paper, we implemented detection of sunset versus 
non-sunset scenes using  support vector machine
\end{abstract}

%
\begin{keywords}
sunset detector,scene classification,support vector machine,image classification
\end{keywords}

%
\section{Introduction}
\label{sec:intro}
For the sunset detector project, we set out to devise an accurate way to classify
whether a given image is a sunset or not using supervised machine learning techniques.

Scene classification has several practical applications. One such use is in consumer
imaging devices to determine the best post-processing steps to deliver a high quality
image. Another would be automated image sorting based on the scene type (e.g., being 
able to classify portraits, nature shots, and so on, to categorize them in an album). 

Scene classification is difficult, even in the binary case that we consider in this
paper. Certain scenes would be very difficult to classify, as they would share the
same characteristics, such as hue and brightness features, as other scenes.

\end{document}
