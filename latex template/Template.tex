% Template for ICME-2010 paper; to be used with:
%          spconf.sty  - ICASSP/ICIP LaTeX style file, and
%          IEEEbib.bst - IEEE bibliography style file.
% --------------------------------------------------------------------------
\documentclass{article}
\usepackage{spconf_ICME,amsmath,epsfig,fancyhdr}
\setlength{\paperwidth}{215.9mm} \setlength{\hoffset}{-9.7mm}
\setlength{\oddsidemargin}{0mm} \setlength{\textwidth}{184.3mm}
\setlength{\columnsep}{6.3mm} \setlength{\marginparsep}{0mm}
\setlength{\marginparwidth}{0mm} \setlength{\paperheight}{279.4mm}
\setlength{\voffset}{-7.4mm} \setlength{\topmargin}{0mm}
\setlength{\headheight}{0mm} \setlength{\headsep}{0mm}
\setlength{\topskip}{0mm} \setlength{\textheight}{235.2mm}
\setlength{\footskip}{12.4mm} \setlength{\parindent}{1pc}


\ICMEfinalcopy % *** Uncomment this line for the final submission

\def\ICMEPaperID{****} % *** Enter the ICME Paper ID here
\def\httilde{\mbox{\tt\raisebox{-.5ex}{\symbol{126}}}}

% Pages are numbered in submission mode, and unnumbered in camera-ready
\ifICMEfinal\pagestyle{empty}\fi


\begin{document}\sloppy

% Example definitions.
% --------------------
\def\x{{\mathbf x}}
\def\L{{\cal L}}


% Title.
% ------
\title{CAMERA-READY GUIDELINES FOR ICME 2010 PROCEEDINGS}
%
% Single address.
% ---------------
\name{Author(s) Name(s)\thanks{Thanks to XYZ agency for funding.}}
\address{Author Affiliation(s) \\
Email: abc@xyz.com}


\maketitle
% insert page header and footer here for IEEE PDF Compliant
\thispagestyle{fancy} \fancyhead{} \lhead{}
\lfoot{978-1-4244-7493-6/10/\$26.00~\copyright2010 IEEE} \cfoot{}
\rfoot{ICME 2010}
\renewcommand{\headrulewidth}{0pt}
\renewcommand{\footrulewidth}{0pt}




%
\begin{abstract}
The abstract should appear at the top of the left-hand column of
text, about 0.5 inch (12 mm) below the title area and no more than
3.125 inches (80 mm) in length.  Leave a 0.5 inch (12 mm) space
between the end of the abstract and the beginning of the main text.
The abstract should contain 2000 characters maximum, and should be
identical to the abstract of the manuscript submitted
electronically. All manuscripts must be in English and printed in
black ink.
\end{abstract}
%
\begin{keywords}
One, two, three, four, five
\end{keywords}
%
\section{Introduction}
\label{sec:intro}

These guidelines include complete descriptions of the fonts,
spacing, and related information for producing your proceedings
manuscripts. Please follow them and if you have any questions,
direct them to Andy W. H. Khong at \verb"andykhong@ntu.edu.sg" or
Clara Lee at \verb"elhlee@ntu.edu.sg" .

\section{Paper Format}
\label{sec:format}

All printed material, including text, illustrations, and charts,
must be kept within a print area of 7 inches (178~mm) wide by 9
inches (229~mm) high. Do not write or print anything outside the
print area. The top margin must be 1 inch (25~mm), except for the
title page, and the left margin must be 0.75 inch (19~mm).  All {\it
text} must be in a two-column format. Columns are to be 3.39 inches
(86~mm) wide, with a 0.24 inch (6~mm) space between them. Text must
be fully justified.

\section{Title Page}
\label{sec:pagestyle}

The paper title (on the first page) should begin 1.38 inches (35 mm)
from the top edge of the page, centered, completely capitalized, and
in Times 14-point, boldface type. Your name and affiliation should
be included for this camera-ready submission.



\subsection{\LaTeX~Users}
If you are using \LaTeX, make sure Line 17 of ``Template.tex'' is
uncommented when generating your camera-ready paper. The authors'
name(s) and affiliation(s) should appear in the camera-ready
submission. In this case simply key in the authors' names and
affiliations in line 40 if you are using \LaTeX.

\subsection{Microsoft Word Users}
If you are using Microsoft Word, please enter your name and
affiliation below the title of your paper.

\section{COPYRIGHT NOTICE}
The appropriate copyright notice should appear at the bottom of the
first page of each paper. For \LaTeX users, you can change this at
Line 48 of the ``Template.tex'' file.

\subsection{Authors employed by the US government}
For papers in which all authors are employed by the US government,
the notice is ``U.S. Government work not protected by U.S.
copyright''.

\subsection{Authors employed by a Crown government}
For papers in which all authors are employed by a Crown government
(U.K., Canada and Australia), the notice is
``978-1-4244-7493-6/10/\$26.00 \copyright2010 Crown''.


\subsection{All other papers}
For all other papers, the notice is ``978-1-4244-7493-6/10/\$26.00
\copyright2010 IEEE''.

\subsection{Copyright Forms} \label{sec:copyright}

You must include your fully completed IEEE copyright release form
when you submit your camera-ready paper. This form must be completed
\emph{electronically} and submitted via the ICME2010 paper
submission system before your paper can be published in the
proceedings. Details of electronic copyright form submission can be
found on the website \verb"http://www.icme2010.org/authorguide.html"
.


\section{TYPE-STYLE AND FONTS}
\label{sec:typestyle}

To achieve the best rendering both in the proceedings and from the
CD-ROM, you are strongly encouraged to use Times New Roman font.  In
addition, this will give the proceedings a more uniform look.  Use a
font that is no smaller than nine point type throughout the paper,
including figure captions.

In nine point type font, capital letters are 2~mm high.  If you use
the smallest point size, there should be no more than 3.2~lines/cm
(8~lines/inch) vertically.  This is a minimum spacing; 2.75~lines/cm
(7~lines/inch) will make the paper much more readable.  Larger type
sizes require correspondingly larger vertical spacing.  Please
adhere to the vertical spacing of this template and do not
double-space your paper. True-Type~1 fonts are preferred.

The first paragraph in each section should not be indented, but all the
following paragraphs within the section should be indented as these paragraphs
demonstrate.

\subsection{Embedding Fonts}
\label{sec:embedFonts}

All submissions must be in PDF. It is important to make sure that
all fonts are embedded. This will ensure that reviewers will be able
to read your manuscript independent of his/her system. You should
check if all your fonts are embedded via Adobe Acrobat under the
following menu ``\emph{File$>$Properties$>$Fonts}.'' If any fonts
are not embedded, start your Acrobat Distiller then
``\emph{Settings$>$Job Options$>$Fonts$>$Embed All Fonts}.''

Alternatively, if you are using Acrobat Professional, go to
``\emph{Edit$>$Preferences$>$Convert to PDF$>$ Microsoft Office
Word$>$ Edit Settings$>$ Edit$>$ Fonts}.'' Then make sure all the
fonts listed in the ``\emph{Font Source}'' dialogue box is added to
the ``\emph{Always Embed}'' dialogue box. Finally click ``\emph{Save
As...}'' and give it a file name. The next time you print any
figures or text to PDF via Microsoft Word or Visio, print them using
the file name you specified. This file name can be specified under
``\emph{Properties}'' in printing settings.



\section{MAJOR HEADINGS}
\label{sec:majhead}

Major headings, for example, ``1. Introduction'', should appear in
all capital letters, bold face if possible, centered in the column,
with one blank line before, and one blank line after. Use a period
(``.'') after the heading number, not a colon. For \LaTeX users,
section numberings will automatically be formatted.

\subsection{Subheadings}
\label{ssec:subhead}

Subheadings should appear in lower case (initial word capitalized) in
boldface.  They should start at the left margin on a separate line.

\subsubsection{Sub-subheadings}
\label{sssec:subsubhead}

Sub-subheadings, as in this paragraph, are strongly discouraged.
However, if you must use them, they should appear in lower case
(initial word capitalized) and start at the left margin on a
separate line, with paragraph text beginning on the following line.
They should be in italics.


\section{LENGTH}
\label{sec:length}

All papers must not exceed \emph{six} pages. Over-length papers will
not be published. Note that all pages must adhere to the margins
described in Section~\ref{sec:format}.


\section{PAPER SUBMISSION AND FILENAME}
\label{sec:paperSub}

All papers must be printed to PDF. This means that you should print
your manuscript to PDF regardless of whether you are using \LaTeX or
Miscrosoft Word Template.

In addition, all camera-ready PDFs \emph{must be} IEEE
Xplore-compatible. To check whether your PDF file is IEEE
Xplore-compatible, please visit \verb"www.pdf-express.org/" . You
can also use the above link to convert your source file(s) to IEEE
Xplore-compatible PDF. The ``Conference ID'' is ``icme10x''.

Please remember to upload the PDF generated from PDF eXpress to the
ICME 2010 submission site. The filename \emph{must be} in the form
of ``xxxx.pdf'' where ``xxxx'' are the four-digit Paper ID.


\section{PAGE NUMBERING}
\label{sec:page}

Please do {\bf not} paginate your paper.  Page numbers, session numbers, and
conference identification will be inserted when the paper is included in the
proceedings.

% Below is an example of how to insert images. Delete the ``\vspace'' line,
% uncomment the preceding line ``\centerline...'' and replace ``imageX.ps''
% with a suitable PostScript file name.
% -------------------------------------------------------------------------
%
\begin{figure}[t]
\begin{minipage}[b]{1.0\linewidth}
  \centering
% \centerline{\epsfig{figure=image1.ps,width=8.5cm}}
  \vspace{2.0cm}
  \centerline{(a) Result 1}\medskip
\end{minipage}
%
\begin{minipage}[b]{.48\linewidth}
  \centering
% \centerline{\epsfig{figure=image3.ps,width=4.0cm}}
  \vspace{1.5cm}
  \centerline{(b) Results 2}\medskip
\end{minipage}
\hfill
\begin{minipage}[b]{0.48\linewidth}
  \centering
% \centerline{\epsfig{figure=image4.ps,width=4.0cm}}
  \vspace{1.5cm}
  \centerline{(c) Result 3}\medskip
\end{minipage}
%
\caption{Example of placing a figure with experimental results.}
\label{fig:res}
\end{figure}

\section{ILLUSTRATIONS, GRAPHS, AND PHOTOGRAPHS}
\label{sec:illust}

Illustrations must appear within the designated margins.  They may
span the two columns.  If possible, position illustrations at the
top of the columns, rather than in the middle or at the bottom.
Caption and number every illustration. All halftone illustrations
must be clear black and white prints.

Since there are many ways, often incompatible, of including images
(e.g., with experimental results) in a \LaTeX document.
Figure~\ref{fig:res} shows you an example of how to do this.


\section{TABLES and EQUATIONS}
\label{sec:tabEqn}

Tables and important equations must be centered in the column.
Table~\ref{tab:cap} shows an example of a table while the equation
\begin{eqnarray}
y &=& ax^2+bx+c
\nonumber
\\
~ &=& (x+p)(x+q)
\end{eqnarray}
shows an example of an equation layout.

\begin{table}[t]
\begin{center}
\caption{Table caption} \label{tab:cap}
\begin{tabular}{|c|c|c|}
  \hline
  % after \\: \hline or \cline{col1-col2} \cline{col3-col4} ...
  Column One & Column Two & Column Three
  \\
  \hline
  Cell 1 & Cell 2 & Cell 3 \\
  Cell 4 & Cell 5 & Cell 6 \\
  \hline
\end{tabular}
\end{center}
\end{table}

Large tables or long equations may span across both columns. Any
table or equation that takes up more than one column width must be
positioned either at the top or at the bottom of the page.



\section{FOOTNOTES}
\label{sec:foot}

Use footnotes sparingly and write them at the bottom of the column
on the page where they are referenced. Use Times New Roman 9-point
type, single-spaced. To help your readers, avoid using footnotes
altogether but include any necessary observations in the text
(within parentheses, if you prefer, as in this sentence).



\section{CITATIONS and REFERENCES}
\label{sec:review}

Citation of your own work is now allowed in this camera-ready
submission. You can now use phrases such as ``\emph{In previous
works [20], we have proposed...}''

List and number all bibliographical references at the end of the
paper. The references can be numbered in alphabetic order or in
order of appearance in the document. When referring to them in the
text, type the corresponding reference number in square brackets as
shown at the end of this sentence. All citations must be adhered to
IEEE format and style. Examples such
as~\cite{Morgan2005},~\cite{cooley65} and~\cite{haykin02} are given
in Section 14.

% References should be produced using the bibtex program from suitable
% BiBTeX files (here: strings, refs, manuals). The IEEEbib.bst bibliography
% style file from IEEE produces unsorted bibliography list.
% -------------------------------------------------------------------------
\bibliographystyle{IEEEbib}
\bibliography{myRef}

\end{document}
